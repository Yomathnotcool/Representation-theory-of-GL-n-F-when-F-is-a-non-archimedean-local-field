\documentclass[12pt,a4paper,english]{article}
\usepackage[a4paper]{geometry}
\usepackage[utf8]{inputenc}
\usepackage[OT2,T1]{fontenc}
\usepackage[keeplastbox]{flushend}


\usepackage{color}
\usepackage{tikz-cd}
\usepackage{appendix}
\usepackage{babel}
\usepackage{dsfont}
\usepackage{amsmath}
\usepackage{amssymb}
\usepackage{amsthm}
\usepackage{stmaryrd}
\usepackage{color}
\usepackage{array}
\usepackage{hyperref}
\usepackage{graphicx}
\usepackage{mathtools}
\usepackage{natbib}
\usepackage[bb=boondox]{mathalfa}
\geometry{top=3cm,bottom=3cm,left=2.5cm,right=2.5cm}
\setlength\parindent{0pt}
\renewcommand{\baselinestretch}{1.3}

\newcommand\restr[2]{{% we make the whole thing an ordinary symbol
  \left.\kern-\nulldelimiterspace % automatically resize the bar with \right
  #1 % the function
  \vphantom{\big|} % pretend it's a little taller at normal size
  \right|_{#2} % this is the delimiter
  }}
  
% definition of the "structure"
\theoremstyle{plain}
\newtheorem{thm}{Theorem}[section]
\newtheorem{lem}[thm]{Lemma}
\newtheorem{prop}[thm]{Proposition}
\newtheorem{coro}[thm]{Corollary}
\newtheorem{claim}{Claim}


\theoremstyle{definition}
\newtheorem{conj}{Conjecture}
\newtheorem{defi}{Definition}
\newtheorem*{example}{Example}
\newtheorem{exercise}{\textbf{\textcolor{red}{Exercise}}}
\newtheorem{step}{Step}

\theoremstyle{remark}

\newtheorem*{rem}{Remark}

% define new control sequence
\newcommand{\homo}{\mathbf{Hom}}
\newcommand{\Max}{\mathbf{Max}}
\newcommand{\spec}{\mathbf{Spec}}


\title{The dimension of the quotient space $E_{H,\theta}$}
\date{\today}
\author{Deng Zhiyuan}


\begin{document}
\maketitle
\newpage

\tableofcontents
\newpage

\begin{abstract}
The main goal of this undergraduate dissertation is to study the dimension of the quotient space $E_{H,\theta}=E/E(H,\theta)$, which is from the theorem 5.16 in \cite{bernstein1976representations}.
\end{abstract}
\newpage

\section{Distributions on $l$-spaces, $l$-groups, and $l$-sheaves and the representation of $l$-group}
\begin{defi}
A topological space $X$ is called an $l-$space if it is Hausdorff, locally compact, and zero-dimensional, i.e. each point has a fundamental system of open compact neighbourhoods.
\end{defi}
\section{Non-degenerate Representations}

 





\newpage
\bibliographystyle{plain}
\bibliography{bib.bib}


\end{document}
